\documentclass[10pt, a4paper]{article}
%\usepackage[utf8]{inputenc}
% especifico márgenes manualmente
\usepackage[paper=a4paper, left=1.5cm, right=1.5cm, bottom=1.5cm, top=3.5cm]{geometry}
% codificación ISO-8859-1
%\usepackage[latin1]{inputenc}
% separación silábica en castellano
\usepackage[spanish]{babel}
\usepackage{framed}


\title{Trabajo Pr\'actico 1}
\author{Eric Brandwein}
\date{Abril 2018}

\begin{document}

\maketitle

% compilar 2 veces para actualizar las referencias
\tableofcontents

\pagebreak
%\newpage


\section{Descripci\'on del problema}

%\begin{framed}
El problema a resolver en este informe será el ya conocido problema de la mochila, o "Knapsack problem", en inglés. Su enunciado dice así: \\
Dado un conjunto de $n$ ítems $S$, cada uno con un tamaño asociado $w_i$ y un beneficio asociado $p_i$, y una mochiula con una capacidad asociada $W$, encontrar el subconjjunto de ítems de $S$ que maximice el beneficio total sin exceder la capacidad de la mochila. Es decir, encontrar $R \subseteq S$ tal que $\sum_{i \in R} p_i$ sea máxima y se cumpla $\sum_{i \in R} w_i \leq W$. \\

%\end{framed}

\end{document}
