\documentclass[10pt, a4paper]{article}
%\usepackage[utf8]{inputenc}
% especifico márgenes manualmente
\usepackage[paper=a4paper]{geometry}
% codificación ISO-8859-1
%\usepackage[latin1]{inputenc}
% separación silábica en castellano
\usepackage[spanish]{babel}
%\usepackage{framed}
%\usepackage{array}
%\usepackage{tabular}

\title{Trabajo Pr\'actico 1}
\author{Eric Brandwein}
\date{Abril 2018}

\begin{document}

\maketitle

% compilar 2 veces para actualizar las referencias
\tableofcontents

\pagebreak
%\newpage


\section{Descripci\'on del problema}

%\begin{framed}
El problema a resolver en este informe será el ya conocido problema de la mochila, o "Knapsack problem", en inglés. Su enunciado dice así: \\
Dado un conjunto de $n$ ítems $S$, cada uno con un tamaño asociado $w_i$ y un beneficio asociado $p_i$, y una mochiula con una capacidad asociada $W$, encontrar el subconjunto de ítems de $S$ que maximice el beneficio total sin exceder la capacidad de la mochila. Es decir, encontrar $R \subseteq S$ tal que $\sum_{i \in R} p_i$ sea máxima y se cumpla $\sum_{i \in R} w_i \leq W$. Asumiremos que todos los valores mencionados son enteros no negativos.\par

Para entender mejor el problema, mostraremos un ejemplo: \par
Imaginemos que poseemos una mochila con capacidad 25, y quye poseemos 5 ítems, cada uno con su tamaño y beneficio asociados de esta forma:
\begin{tabular}{c|c|c}
	ítem & tamaño & beneficio \\ \hline
	1 & 10 & 5 \\
	2 & 15 & 4 \\
	3 & 5 & 13 \\
	4 & 10 & 8 \\
	5 & 5 & 8
\end{tabular}
La combinación de ítems que maximizará el beneficio sin pasarse de la capacidad de la mochila será el conjunto $\{3, 4, 5\}$, que juntos suman un tamaño de $5 + 10 + 5 = 20$ y un beneficio de $13 + 8 + 8 = 29$. Cualquier otra combinación de los ítems que poseemos o no entrará en la mochila por tener un tamaño total mayor a la capacidad, o tendrá un beneficio total menor o igual al beneficio alcanzado con la solución dada. \par


%\end{framed}
\end{document}
